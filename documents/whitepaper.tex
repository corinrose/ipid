\documentclass{article}

\author{Ian Leinbach and Corin Rose}
\title{Inter-Planetary ID}

\begin{document}

\maketitle

\begin{center}
\textbf{Abstract} 
\end{center}

A universally agreed upon format for defining what data constitutes an ID would allow for interoperability between services. If this format is based on IPLD and represented as hash-linked data, we can leverage IPFS to store and share this information in a distributed way and let users maintain control of their data. By building off of amazing work done in distributed systems, we can create this format as a way of standardizing how a human exists on the Internet, in much the same way that, for example, Unicode standardized the way we represent arbitrarily complex symbols with data.  

\section{Introduction}

\section{Format}

\subsection{Profile}

\subsection{Extensions}

\section{Reference Implementation}

\subsection{Blog Extension}

\subsection{Profile Mirroring}

\section{Conclusion}

\end{document}
